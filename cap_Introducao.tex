\chapter{Introdução} \label{ch:intro}

% Resumo opcional. Comentar se não usar.
%\resumodocapitulo{Resumo opcional}

\section{Contextualização}

O uso de robôs vêm crescendo ao longo dos anos para a execução de diversas tarefas.

Muito embora ainda exista dentro do imaginário popular a visão da robótica como grande vilão no roubo de empregos, o maior esforço recente é em trabalhar na cooperação entre máquinas e pessoas como forma de aliar a alta capacidade de repetibilidade e precisão dos robôs junto a habilidade humana de rápida aprendizagem na hora de lidar com problemas complexos.

Como forma de minimizar o risco de lesão em caso de eventual impacto, algumas modificações têm sido propostas como introdução de atuadores série elásticos e braços robóticos mais leves. Desta forma a inércia do robô é reduzida e permite a complacência no impacto sem impor riscos de dano ao robô ou a pessoa. No entanto o controle do robô acaba sendo menos preciso e acaba limitando a aplicação das tarefas.

% Robótica Cooperativa
% https://medium.com/@abhasvc/ais-threat-to-society-is-scarier-than-trump-ff7e9d42ea74
% https://www.hhs.se/contentassets/c8f677a0c9974bde950e2cec2edc51a1/substitution-of-labor-final.pdf

\section{Definição do problema}

No Laboratório de Automação e Robótica ( LARA - UnB ) encontra-se disponível o braço robótico Meka A2 fornecido pela Meka Robotics. Para simplificar será referênciado apenas por Meka ao longo deste trabalho. O Meka é um braço complacente antropomórfico composto por 7 juntas e uma garra desenvolvido para pesquisas na área de interação com pessoas. Cada uma das juntas possui um atuador série elástico composto de um motor brushless e um redução mecânica por onda de deformação.

Em trabalhos anteriores foi observado um desempenho ineficiente nos controles cinemáticos implementados em contraste ao resultado observado em outras plataformas robóticas e em simulação. Tal arquitetura não encontra-se devidamente documentada e detalhada. Desta forma, neste trabalho, propõe-se realizar uma caracterização mais detalhada desta arquitetura de controle para compreender suas implicações no desempenho dos controles cinemáticos testados e propor melhorias que resultem em trajetórias mais precisas levando as características próprias do sistema.

\section{Metodologia}

\subsection{Trabalhos Anteriores}

Inicialmente foi feito um estudo sobre os trabalho anteriores desenvolvidos na plataforma no Laboratório de Automação e Robótica ( LARA - UnB ). Alguns destes trabalhos puderam ser acompanhados ao longo da execução como o caso do trabalho desenvolvido por Marcos Pereira e pelo Rafael Koji de modo a facilitar a transmissão do conhecimento relacionado as contribuições de cada um ao projeto bem como presenciar as dificuldades relacionadas ao controle do braço robótico.

\subsection{Metodologia de Investigação}

Após o estudo preliminar foi adotado uma metodologia de investigação baseada em ciclos compostos por 3 etapas, descritas a seguir. Estas etapas são baseadas nos princípios de desenvolvimento ágil e têm como objetivo acelerar e otimizar os esforços.

% Diagrama das etapas
\begin{enumerate}
    \item Análise e Formulação de Hipóteses
    \item Formulação de Testes para as Hipóteses
    \item Avaliação de Hipóteses ( Testes Experimentais e Estudo teórico )
\end{enumerate}

Cada ciclo possui duração variável de acordo com o nível de aprofundamento necessário para satisfazer os objetivos levantados para cada momento do projeto. O objetivo central é ao final de cada ciclo trazer algum aprimoramento quanto ao entendimento do sistema ou ainda quanto ao comportamento final na realização da tarefa.

No inicio de cada ciclo é feito um estudo do sistema através da modelagem do sistema e da definição de uma métrica para avaliar o comportamento. Então o comportamento do sistema real é comparado com o modelo adotado e as expectativas de comportamento. Os objetivos são traduzidos em uma métrica, compondo um conjunto de indicadores que permita avaliar se tarefa foi executada e como foi o processo de execução. A cada novo aspecto avaliado podem ser introduzidos novos indicadores em conjunto aos antigos ou modificados. 

Com base nisto são levantadas hipóteses quanto ao comportamento quanto ao fato de satisfazer ou não o modelo adotado e possíveis condutas para aprimorar os resultados dentro da métrica proposta. Para cada uma destas condutas é levantado um teste de verificação que pode incluir consulta bibliográfica, simulações a partir do modelo e experimentos com o sistema. Os resultados desta etapa são então levados para a etapa de observação e confrontados novamente e assim o ciclo se repete.

Resultados positivos são incorporados e resultados negativos avaliados quanto a possíveis correções. Tudo é registrado para permitir futuras avaliações. O resultado de cada dia foi feito um registro das atividades e todo material produzido foi colocado no GitHub de forma a permitir o uso futuro na validação do comportamento após modificações futuras.

Desta forma, os resultados da investigação ao longo do período deste trabalho compõe os capítulos () e () aonde são apresentados os aspectos técnicos de cada componente analisado e resultados dos testes efetuados. 

% Estudo em malha fechada
