\chapter{Introdução} \label{ch:intro}

% Resumo opcional. Comentar se não usar.
%\resumodocapitulo{Resumo opcional}

\section{Contextualização}

% Robótica Cooperativa
% https://medium.com/@abhasvc/ais-threat-to-society-is-scarier-than-trump-ff7e9d42ea74
% https://www.hhs.se/contentassets/c8f677a0c9974bde950e2cec2edc51a1/substitution-of-labor-final.pdf

\section{Definição do problema}



\section{Objetivos do projeto}

Este trabalho têm como objetivo investi

\section{Metodologia}

\subsection{Revisão Bibliográfica}

\subsection{Trabalhos Anteriores}

\subsection{Metodologia de Investigação}

A partir das premissas originais do projeto em conjunto com os princípios de desenvolvimento ágil foi adotado como metodologia de investigação baseada em ciclos compostos por 3 etapas:

% Diagrama das etapas

\begin{enumerate}
    \item Análise e Formulação de Hipóteses
    \item Formulação de Testes para as Hipóteses
    \item Avaliação de Hipóteses ( Testes Experimentais e Estudo teórico )
\end{enumerate}

Este ciclos tiveram duração variável de acordo com o nível de aprofundamento necessário para satisfazer os objetivos levantados para cada momento do projeto. O objetivo central é ao final de cada ciclo trazer algum aprimoramento quanto ao entendimento do sistema ou ainda quanto ao comportamento final quanto a realização da tarefa.

No inicio de cada ciclo foi feito um estudo do sistema através da modelagem do sistema e da definição de uma métrica para avaliar o comportamento. Então o comportamento do sistema real é comparado com o modelo adotado e as expectativas de comportamento.

Os objetivos são traduzidos em uma métrica, compondo um conjunto de indicadores que permita avaliar se tarefa foi executada e como foi o processo de execução. A cada novo aspecto avaliado podem ser introduzidos novos indicadores em conjunto aos antigos ou modificados. 

Com base nisto foram levantadas hipóteses quanto ao comportamento quanto ao fato de satisfazer ou não o modelo adotado e possíveis condutas para aprimorar os resultados dentro da métrica proposta. Para cada uma destas condutas é levantado um teste de verificação que pode incluir consulta bibliográfica, simulações a partir do modelo e experimentos com o sistema. Os resultados desta etapa são então levados para a etapa de observação e confrontados novamente e assim o ciclo se repete.

Resultados positivos são incorporados e resultados negativos são registrados para avaliações. O resultado de cada dia foi registrado em um log de atividades e todo material produzido foi colocado no github de forma a permitir o uso futuro na validação do comportamento após modificações futuras.

% Estudo em malha fechada

\subsection{E}
