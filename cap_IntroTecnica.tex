\chapter{Fundamentos Técnicos}

\section{Meka}

%% Descrição Meka
Para o desenvolvimento deste trabalho foi utilizado o manipulador robótico Meka A2. Sendo composto por 8 juntas com atuadores elásticos. É um braço desenvolvido para pesquisa em tarefas em interação com pessoas de maneira segura em razão do baixo peso e da capacidade de ceder em contato com forças externas. 

\subsection{Atuadores Elásticos}

Cada uma das juntas do Meka A2 utiliza um atuador elástico composto por pelo motor brushless ### em conjunto com uma redução baseada em engrenamento por onda de deformação, comercialmente denominado \textit{Harmonic Drive}.

O uso de redução permite que uma alta velocidade do motor seja traduzida em um alto torque gerando uma grande inércia. Assim, quando ocorre uma colisão muita energia é transmitida ao objeto de contato bem como ao dente da engrenagem de saída, resultando internamente em um fratura no mecanismo da junta do robô. Ao se colocar um elemento elástico como uma mola, parte desta energia é absorvida e distribuída gradualmente reduzindo assim a possibilidade de fratura. 

De igual maneira esta alta inercia representa um risco na execução de tarefas em conjunto com pessoa, pois um impacto nesta situação pode causar grande danos.

Este tipo de atuador com é denominado, atuador série elástico e é constituído basicamente por um motor, um sistema de redução e um elemento elástico dispostos em série com a carga. Como ilustrado no diagrama ## .

%% Diagrama

No meka é utilizado um motor brushless ### permitindo alta velocidade. Para o elemento de redução é usado um sistema de engrenagem por onda de deformação, comercialmente denominado \textit{Harmonic Drive}.

\subsection{Sensores}

Para cada junta são medidas as grandezas a posição e velocidade angular e o torque. Para tal é utilizado dois sensores: um encoder associado diretamente a junta e um sensor de corrente associado ao motor brushless. A posição de cada junta é avaliada a partir do enconder ()

O torque é avaliado de maneira indireta através de sensores de corrente.
% Porque um motor puxa mais corrente quando tẽm que fazer mais esforço ( estudo do ponto de vista de controle )

\subsection{DSP Control Boards}

Cada uma das juntas é controlada por uma placa DSP desenvolvida pela Meka Robotics. Estas possuem uma interface EtherCat e um controlador interno para acionamento dos motores de cada junta, bem como o controle de rigidez.

\subsection{}

\subsection{ROS}

Robotics Operating System, ROS é um framework voltado para a robótica. 

\subsection{PyKDL}

\section{Teoria}

\subsection{Controle}

\subsection{Controle PID}

\subsection{Quaternions Duais}

\section{Trabalho}

\subsection{Estudo controladores API Python}

\subsection{Estudo controladores DQ}

\subsection{Estudo controladores}

