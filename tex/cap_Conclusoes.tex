\chapter{Conclusão} \label{ch:conclusoes}

\resumodocapitulo{``"What we need is more people specialize in the impossible '' (Theodore Roethke)}

\section{Perspectivas Futuras}

% Introdução 
% - Cobots
Neste trabalho foi apresentado um detalhamento maior da arquitetura de controle implementada no Meka permitindo conhecer os fundamentos das tecnologias envolvidas no desenvolvimento de manipuladores robóticos cooperativos bem como vivenciar os principais desafios em manter um manipulador robótico seguro e um desempenho razoável na execução de alguma tarefa. Foi buscado ir a origem inicial de cada problema, tanto do ponto de vista histórico como na análise de cada componente utilizado. Robótica colaborativa ainda representa um tema com bastante espaço de pesquisa e trabalho pois robôs estão cada vez mais acessíveis devido a redução no preço dos computadores, o uso de sistemas série elásticos para controle preciso de torque e bem como a disponibilidade de soluções open-source entre outras tecnologias entre outros fatores. Soluções robóticas sempre estiveram no plano de tecnologias de alto padrão e distantes da realidade da maioria porém têm sido crescente os esforços na difusão.

% Trabalhos Futuros Controle + Metodologia
Apesar de constituir uma plataforma mais acadêmica, o Meka ainda pode ser amplamente utilizado para estudos na área de controle e principalmente para interação com pessoas ou outros robôs. Para trabalhos futuros na plataforma é sugerido a identificação do modelo dos atuadores série elástico disponíveis na plataforma visando a implementação de um controle de juntas mais preciso. No intuito de apresentar a modelagem completa do braço, uma vez que esta não está disponível. Também é apresentado a possibilidade de estudo em controle multivariado a partir de espaço de estados para permitir explorar os efeitos das não linearidade em composição com o modelo cinemático do robô, gerando uma solução que controle ao mesmo a cinemática e dinâmica do movimento. Além de filtros mais sofisticados de forma a permitir o uso da informação do sensor de força em conjunto com os demais sensores. Os conceitos implementados na plataforma podem também compor partes de outros projetos como explorado por outras universidades que passaram a produzir seus próprios atuadores, braços e plataformas robóticas.

%Com base nos resultados obtidos, observou-se o controle implementado em ROS está sendo executado com um frequência maior que a resposta das camadas inferiores subsequentes de controle. Este fator pode estar causando uma instabilidade no controle, uma vez que o erro acumulado durante o tempo que o robô permanece parado até o ocorrer a resposta é compensado com um esforço maior de controle, percebido nos registros dos experimentos como saltos periódicos na velocidade. Estes efeitos eram percebidos com maior nitidez no eixo vertical, uma vez que a maior perturbação ao sistema se dava pelo efeito da ação do campo gravitacional.

%Como proposta, serão efetuados ao longo do trabalho a continuidade do estudo da plataforma em conjunto com a identificação da planta completa. Por se tratar de um sistema complexo que envolve diversos sub-sistemas atuando em conjunto, este estudo será pautado as principais influências de cada componente no controle final da posição do efetuador.

%Também serão analisado um comparativo das estratégias de controle por velocidade e por posição a partir da descrição cinemática implementada em Quatérnions Duais. O controle por velocidade apresenta uma resposta mais rápida que os controles de velocidade por retirar a necessidade e esperar a posição final ser atingida e eliminaria a necessidade do passo de integração por parte de alguns dos controladores implementados. Ao passo que poderia permitir o uso direto de controladores mais baixos, uma vez que os motores Brushless possuem os controle de torque e velocidade que são convertidos em posição por meio da placa DSP.

%Ao final, o objetivo é trazer desempenho melhor em termos de precisão no controle do Meka e facilitar o uso deste no laboratório em projetos futuros por meio do aprofundamento na documentação e estudo iniciados em trabalhos anteriores. Em conjunto é apresentado um detalhamento das sistemas empregados para inspirar o uso em projetos futuros de sistemas robóticos desenvolvidos pelo laboratório, uma vez soluções similares baseadas em atuadores série elásticos estão presentes muitos projetos recentes.
