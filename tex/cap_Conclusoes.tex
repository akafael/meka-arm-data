\chapter{Conclusão} \label{ch:conclusoes}

% "I never think of the future - it comes soon enough." - Albert Einstein

\section{Perspectivas Futuras}

Com base nos resultados obtidos, observou-se o controle implementado em ROS está sendo executado com um frequência maior que a resposta das camadas inferiores subsequentes de controle. Este fator pode estar causando uma instabilidade no controle, uma vez que o erro acumulado durante o tempo que o robô permanece parado até o ocorrer a resposta é compensado com um esforço maior de controle, percebido nos registros dos experimentos como saltos periódicos na velocidade. Estes efeitos eram percebidos com maior nitidez no eixo vertical, uma vez que a maior perturbação ao sistema se dava pelo efeito da ação do campo gravitacional.

Como proposta, serão efetuados ao longo do trabalho a continuidade do estudo da plataforma em conjunto com a identificação da planta completa. Por se tratar de um sistema complexo que envolve diversos sub-sistemas atuando em conjunto, este estudo será pautado as principais influências de cada componente no controle final da posição do efetuador.

Também serão analisado um comparativo das estratégias de controle por velocidade e por posição a partir da descrição cinemática implementada em Quatérnions Duais. O controle por velocidade apresenta uma resposta mais rápida que os controles de velocidade por retirar a necessidade e esperar a posição final ser atingida e eliminaria a necessidade do passo de integração por parte de alguns dos controladores implementados. Ao passo que poderia permitir o uso direto de controladores mais baixos, uma vez que os motores Brushless possuem os controle de torque e velocidade que são convertidos em posição por meio da placa DSP.

Ao final, o objetivo é trazer desempenho melhor em termos de precisão no controle do Meka e facilitar o uso deste no laboratório em projetos futuros por meio do aprofundamento na documentação e estudo iniciados em trabalhos anteriores. Em conjunto é apresentado um detalhamento das sistemas empregados para inspirar o uso em projetos futuros de sistemas robóticos desenvolvidos pelo laboratório, uma vez soluções similares baseadas em atuadores série elásticos estão presentes muitos projetos recentes.