\chapter{Conclusão} \label{ch:conclusoes}

\resumodocapitulo{``"What we need is more people specialize in the impossible '' (Theodore Roethke)}

\section{Perspectivas Futuras}

% Introdução 
% - Cobots
Neste trabalho foi apresentado um detalhamento maior da arquitetura de controle implementada no Meka permitindo conhecer os fundamentos das tecnologias envolvidas no desenvolvimento de manipuladores robóticos cooperativos bem como vivenciar os principais desafios em manter um manipulador robótico seguro e um desempenho razoável na execução de alguma tarefa. Foi buscado ir a origem inicial de cada problema, tanto do ponto de vista histórico como de cada componente utilizado. O redução no preço dos computadores, o uso de sistemas série elásticos para controle preciso de torque e bem como a disponibilidade de soluções open-source entre outras tecnologias permitiram que manipuladores robóticos pudessem alcançar mais pessoas. Soluções robóticas sempre estiveram no plano de tecnologias de alto padrão e distantes da realidade da maioria porém têm sido crescente os esforços na difusão.

% Fundamentação
% - Qualidade do HW
O Meka foi implementado com componentes de alto desempenho seguindo a filosofia do mercado na sua época de lançamento. Embora hoje, outras tecnologias estão sendo incorporadas ao projeto de robôs complacentes como o uso de sistemas freios, esta plataforma ainda representa uma ótima plataforma para o estudo de controladores dinâmicos bem como cada uma dos componentes envolvidos. Como exemplo, o uso de atuadores série elásticos, embora estejam implementados no Meka com motores e sistemas de transmissão de ponta, podem também ser incorporados em outros projetos como solução aos problemas de instabilidade do controle na interação com superfícies rígidas em aplicações para locomoção. Ou ainda em aplicação de forças na área de fisioterapia.

% Desenvolvimento
No que tange ao sistema de controle implementado, foi percebido que nos trabalhos anteriores foi alcançado o limite de desempenho dos controladores de baixo atuais do sistema M3. No entanto, dado que o sistema de controle é completamente open-source e possui um sistema embarcado capaz de uma desempenho maior ainda existe um enorme potencial que pode ser explorado para uso em trabalhos futuros. Uma vez que as soluções implementadas incorporam técnicas de filtragem e controle dinâmico que podem ser aprimoradas.

% Resultados
% - Juntas
Os controladores de posição de juntas foram ajustados para responderem de forma similar para todas as juntas. Embora esta seja uma estratégia que permite que a resposta controle seja similar e facilite a avaliação dos controladores cinemáticos, na prática o comportamento o comportamento de cada junta é bem diferente. Para um ganho de desempenho, pode ser explorado a possibilidade de controladores ajustados para respostas diferente para cada uma das juntas. Em conjunto com as características particulares da geometria do braço, um desempenho melhor pode ser atingido, por permitir as juntas do ombro operarem com uma resposta mais rápida.

A técnica atualmente implementada para compensação da gravidade usa apenas a posição das juntas para uma gerar um torque em feedforward a partir do modelo cinemático e dinâmico do robô. Outras técnicas podem ser utilizadas para que as velocidade final e o torque possam também ajustadas. Também foi notado que o acoplamento entre a posição e a orientação introduzido por quatérnions duais pode ser explorado a partir das características das juntas do robô, uma vez que a composição do ombro e do pulso podem atuar como dois guibal em oposição permitindo um estudo maior da dinâmica da rotação no movimento do efetuador.

% Trabalhos Futuros Controle + Metodologia
Para trabalhos futuros na plataforma é sugerido a identificação do modelo dos atuadores série elástico disponíveis na plataforma visando a implementação de um controle de juntas mais preciso. No intuito de apresentar a modelagem completa do braço, uma vez que esta não está disponível. Também é apresentado a possibilidade de estudo em controle multivariado a partir de espaço de estados para permitir explorar os efeitos das não linearidade em composição com o modelo cinemático do robô, gerando uma solução que controle ao mesmo a cinemática e dinâmica do movimento. Os conceitos implementados na plataforma podem também compor partes de outros projetos com explorado por outras universidades que passaram a produzir seus próprios atuadores, braços e plataformas robóticas.

%Com base nos resultados obtidos, observou-se o controle implementado em ROS está sendo executado com um frequência maior que a resposta das camadas inferiores subsequentes de controle. Este fator pode estar causando uma instabilidade no controle, uma vez que o erro acumulado durante o tempo que o robô permanece parado até o ocorrer a resposta é compensado com um esforço maior de controle, percebido nos registros dos experimentos como saltos periódicos na velocidade. Estes efeitos eram percebidos com maior nitidez no eixo vertical, uma vez que a maior perturbação ao sistema se dava pelo efeito da ação do campo gravitacional.

%Como proposta, serão efetuados ao longo do trabalho a continuidade do estudo da plataforma em conjunto com a identificação da planta completa. Por se tratar de um sistema complexo que envolve diversos sub-sistemas atuando em conjunto, este estudo será pautado as principais influências de cada componente no controle final da posição do efetuador.

%Também serão analisado um comparativo das estratégias de controle por velocidade e por posição a partir da descrição cinemática implementada em Quatérnions Duais. O controle por velocidade apresenta uma resposta mais rápida que os controles de velocidade por retirar a necessidade e esperar a posição final ser atingida e eliminaria a necessidade do passo de integração por parte de alguns dos controladores implementados. Ao passo que poderia permitir o uso direto de controladores mais baixos, uma vez que os motores Brushless possuem os controle de torque e velocidade que são convertidos em posição por meio da placa DSP.

%Ao final, o objetivo é trazer desempenho melhor em termos de precisão no controle do Meka e facilitar o uso deste no laboratório em projetos futuros por meio do aprofundamento na documentação e estudo iniciados em trabalhos anteriores. Em conjunto é apresentado um detalhamento das sistemas empregados para inspirar o uso em projetos futuros de sistemas robóticos desenvolvidos pelo laboratório, uma vez soluções similares baseadas em atuadores série elásticos estão presentes muitos projetos recentes.