%% LyX 2.1.4 created this file.  For more info, see http://www.lyx.org/.
%% Do not edit unless you really know what you are doing.
\documentclass[a4paper,oneside,brazil,11pt,a4paper,openright,titlepage,usenames,dvipsnames]{book}
\usepackage[utf8]{inputenc}
\usepackage[T1]{fontenc}
\usepackage[brazilian]{babel}
\usepackage{lmodern}
%\usepackage{times}
\setcounter{secnumdepth}{3}
\setcounter{tocdepth}{3}
\usepackage{array}
\usepackage{verbatim}
\usepackage{calc}
\usepackage{textcomp}
\usepackage{amssymb}
\usepackage[pdf]{graphviz}

\makeatletter

%%%%%%%%%%%%%%%%%%%%%%%%%%%%%% LyX specific LaTeX commands.
\pdfpageheight\paperheight
\pdfpagewidth\paperwidth

%% Because html converters don't know tabularnewline
\providecommand{\tabularnewline}{\\}

%%%%%%%%%%%%%%%%%%%%%%%%%%%%%% User specified LaTeX commands.
% Classe alternativa, apropriada para impressão frente-verso. Inclui páginas em branco
% de forma que capítulos sempre tenham início na página à direita:
% \documentclass[11pt,a4paper,openright,titlepage]{book}

% Pacotes
\usepackage[T1]{fontenc}
\usepackage[brazilian]{babel}
\usepackage{epsfig}
\usepackage{subfigure}
\usepackage{amsfonts}
\usepackage{amsmath}
\usepackage[thmmarks,amsmath]{ntheorem}%\usepackage{amsthm}
\usepackage{boxedminipage}
\usepackage{geometry}
\usepackage{theorem}
\usepackage{fancybox}
\usepackage{fancyhdr}
\usepackage{ifthen}
\usepackage{url}
\usepackage{afterpage}
\usepackage{color}
\usepackage{colortbl}
\usepackage{rotating}
\usepackage{makeidx}
\usepackage{indentfirst}
\usepackage{float}
\usepackage{caption}
\usepackage{subcaption} % permite o uso de multiplas figuras com legenda (ambiente subfigure)
% Pacotes para adição de figuras do inkscape
\usepackage{graphicx}
\usepackage{lastpage}
\usepackage{import}

% Escolher um dos seguintes formatos:
\usepackage{ft2unb} % segue padrão de fontes do LaTeX

\makeindex

\makeatother

\usepackage{babel}
\begin{document}
\setcounter{secnumdepth}{3}
\setcounter{tocdepth}{2}
\pagestyle{empty}

\grau{Engenheiro de Controle e Automação}

\tipodemonografia{TRABALHO DE GRADUAÇÃO}

\begin{comment}
Título
Estudo e caracterização de arquitetura de controle de braço robótico complacente
\end{comment}

\titulolinhai{\MakeUppercase{Caracterização e implementação de melhorias}}

\titulolinhaii{\MakeUppercase{em arquitetura de controle de braço}}

\titulolinhaiii{\MakeUppercase{robótico complacente}}

\titulolinhaiv{}

% Autores. Basta retirar o texto totalmente caso não haja um determinado autor.
\autori{Rafael Lima}
\autorii{}
\autoriii{}

% Membros da banca. Basta retirar o texto totalmente caso não haja um determinado membro da banca.
\membrodabancai{Prof. Geovany Araujo Borges, ENE/UnB}
\membrodabancaifuncao{Orientador}
\membrodabancaii{Prof. Mariana Bernardes, FGA/UnB}
\membrodabancaiifuncao{Co-orientador}
\membrodabancaiii{Prof. João Yoshiyuki Ishihara, ENE/UnB}
\membrodabancaiiifuncao{Examinador interno}
\membrodabancaiv{Prof. Hugo Tadashi Muniz Kussaba, ENE/UnB}
\membrodabancaivfuncao{Examinador interno}
\membrodabancav{}
\membrodabancavfuncao{}

% Data da Defesa
\mes{Dezembro}
\ano{\the\year}

% Capas
\capaprincipal
\capaassinaturas

% Ficha Catalográfica 
\noindent \textbf{FICHA CATALOGRÁFICA}

\noindent %
\fbox{\begin{minipage}[t]{1\columnwidth}%
LIMA, RAFAEL

Estudo e caracterização de arquitetura de controle em braço robótico complacente,

\medskip{}


{[}Distrito Federal{]} 2018.

\medskip{}


x, \pageref{LastPage}p., 297 mm (FT/UnB, Engenheiro, Controle e Automação, 2018).
Trabalho de Graduação \textendash{} Universidade de Brasília.Faculdade
de Tecnologia.

\medskip{}


1. Robótica Colaborativa\hfill{}2.Manipuladores Robóticos\hfill{}

3. Controle

\medskip{}


I. Mecatrônica/FT/UnB\hfill{}II. Título (Série)\hfill{}

%
\end{minipage}}

\noindent \medskip{}


\noindent \textbf{REFERÊNCIA BIBLIOGRÁFICA}

LIMA, RAFAEL, (2018). Estudo e caracterização de arquitetura de controle em braço robótico complacente. Trabalho de Graduação
em Engenharia de Controle e Automação, Publicação FT.TG-$n^{\circ}022$,
Faculdade de Tecnologia, Universidade de Brasília, Brasília, DF, \pageref{LastPage}p.

\noindent \bigskip{}


\noindent \textbf{CESSÃO DE DIREITOS}

\noindent AUTOR: \autorinome

Estudo e caracterização de arquitetura de controle em braço robótico complacente.

\noindent \medskip{}


\noindent GRAU: Engenheiro\hfill{}ANO: 2018\hfill{}

\noindent \medskip{}


É concedida à Universidade de Brasília permissão para reproduzir cópias
deste Trabalho de Graduação e para emprestar ou vender tais cópias
somente para propósitos acadêmicos e científicos. O autor reserva
outros direitos de publicação e nenhuma parte desse Trabalho de Graduação
pode ser reproduzida sem autorização por escrito do autor.

\noindent \bigskip{}


\noindent \rule[0.5ex]{1\columnwidth}{1pt}

\noindent \autorinome

email: raffaellimma@gmail.com
%\noindent Rua dos Bobos, nº 0, Bairro Feliz.

%\noindent 71000-000 Brasília \textendash{} DF \textendash{} Brasil.


% Dedicatória -------------------------------------------------------
\frontmatter

% Dedicatória primeiro autor
\dedicatoriaautori{Dedico este trabalho a todos que queriam conhecer um pouquinho deste vasto mundo da tecnologia}

% Texto de dedicatória do segundo autor. Caso não tenha um segundo autor, este texto não será mostrado
%\dedicatoriaautorii{Dedicatória do autor 2}

% Texto de dedicatória do terceiro autor. Caso não tenha um segundo autor, este texto não será mostrado
%\dedicatoriaautoriii{Dedicatória do autor 3}

%\dedicatoria

% Agradecimentos --------------------------------------------------------

\agradecimentosautori{
% Intro
% Absolutamente nada nesta vida se faz só. Para chegar até aqui devo a tantos que citando teria mais páginas que o TG.

% Familia
%Agradeço primeiramente a meus pais, Franscico e Zely pois devo tudo a eles, desde minha vida a cada aspecto construído nela. A meus irmãos Lorrane e Mateus, pela imensa parceria em qualquer situação. A minha namorada Lia, que me ajudou ao longo de tantas vezes que esqueci completamente de mim enquanto envolvido com alguma tarefa do curso. A minha família, tios, primos e avós.

% Amigos: Ciro, Pedro
% Turma 27:
% Aos colegas da turma 27 de engenharia mecatrônica: Ataías, David, Marina, Marcos, Marco Emilio,,, Jessé, Rodrigo, Caixeta, De Hong, Fugita, Carioca, Gabriel, Caveira, Guilherme, Victor, Artur , Cris, Iuri, Fred, Pedro, José, Luís, Caio, Jéssica e Ícaro.
% Lara2: Eric, Miguel, Gabriel Moisés, Bruno, Breno,
% Lara1: Cláudia, David, Roberto, Thiago, George
% I3E: Pâmela, Oniran, Josh, ...
% Amigos: André, Arthur
% Amigos Curso: A tantos amigos no curso, 

% Professores Matérias pesadas
%Aos professores Benilton e Felipe da sala de recurso, onde comecei meu aprendizado sobre computadores e robótica. Dentro da universidade aos professores Clod Wagner, Célius Magalhães, Rathie Puspa, Arthur Vicentini, Fábio Menezes, Marcos Lamar, José Maurício, Renato Borges, Rafael Shayani, por terem me ajudado a construir uma base sólida ao longo do curso.

% Funcionários
% Valter, Carlos, 

%Ao professor Yuri Dumaresq e a professora Carla Koike, dois enormes preceptores ao longo de quase todo tempo na UnB. Aos professores do Lara, prof Antônio, a qual devo a honra do convite para começar a fazer parte do Lara e a participação no projeto da prótese, ..., prof. Adolfo,...

%Aos professores no intercâmbio, Holly, Ilene, Dean, Laura, Ammar e Ana, aprendi tanto com cada um. Em especial Mohammad Mehrmohammad, que me acompanhou durante todo tempo e no laboratório de Ultrassom, jamais conheci alguém com o coração tão bondoso e tão motivado a fazer a diferença.

%E por fim em especial para a prof. Mariana, que propôs a temática e me acompanhou durante toda jornada desde o trabalho no Cyton, passando por quatérnions duais até atuadores série elásticos. E ao prof. Geovany por me ajudar na orientação deste trabalho com diversas dicas extremamente precisas em vários momentos que eu estava completamente perdido.

%A cada
}

%\agradecimentosautorii{A inclusão desta seção de agradecimentos é opcional e fica à critério do(s) autor(es), que caso deseje(em) inclui-la deverá(ão) utilizar este espaço, seguindo esta formatação.}

%\agradecimentosautoriii{A inclusão desta seção de agradecimentos é opcional e fica à critério do(s) autor(es), que caso deseje(em) inclui-la deverá(ão) utilizar este espaço, seguindo esta formatação.}

\agradecimentos

% Resumo ------------------------------------------------------------

\resumo{resumo}{Resumo!

\medskip{}

%Palavras Chave: bla, ble, bli

}\vspace*{2cm}

\resumo{Abstract}{Abstract, in English ofc!

\medskip{}

%Keywords: bla, ble, bli

}

% Sumário, Listas de figuras e listas de tabelas ---------------------

\sumario
\listadefiguras
\listadetabelas
 Listas de símbolos
%TCIDATA{LaTeXparent=0,0,these.tex}


%\chapter*{\setfontarial\mdseries LISTA DE SÍMBOLOS} % se usar ft1unb.sty, descomente esta linha



\chapter*{LISTA DE SÍMBOLOS}

% se usar ft2unb.sty, descomente esta linha

\subsection*{Símbolos Latinos}

\begin{tabular}{p{0.1\textwidth}p{0.63\textwidth}>{\PreserveBacklash\raggedleft}p{0.15\textwidth}}
$v$  & Velocidade linear  & {[}m/s{]}\tabularnewline\\
$a$  & Aceleração linear  & {[}m/s{]}\tabularnewline\\
$F$  & Força  & {[}m/s{]}\tabularnewline
\end{tabular}


\subsection*{Símbolos Gregos}

\begin{tabular}{p{0.1\textwidth}p{0.63\textwidth}>{\PreserveBacklash\raggedleft}p{0.15\textwidth}}
$\omega$ & Velocidade angular & {[}rad/s{]}\tabularnewline
$\alpha$ & Aceleração angular & {[}rad/s{]}\tabularnewline
$\tau$ & Torque & {[}rad/s{]}\tabularnewline
\end{tabular}


\subsection*{Grupos Adimensionais}

\begin{tabular}{p{0.1\textwidth}p{0.8\textwidth}}
i, k & Contador\tabularnewline
\end{tabular}


\subsection*{Subscritos}

\begin{tabular}{p{0.1\textwidth}p{0.8\textwidth}}
$ref$  & referência \tabularnewline
$fer$  & ferramenta \tabularnewline
$sis$  & sistema \tabularnewline
$des$  & desejado\tabularnewline
\end{tabular}


\subsection*{Sobrescritos}

\begin{tabular}{p{0.1\textwidth}p{0.8\textwidth}}
$\cdot$  & Variação temporal \tabularnewline
$-$  & Valor médio \tabularnewline
$^{}$  & Vetor Unitário \tabularnewline
\end{tabular}


\subsection*{Siglas}

\begin{tabular}{p{0.1\textwidth}p{0.8\textwidth}}
%PCI  & \textit{Peripheral Component Interconnect}\tabularnewline
%CPU & Unidade Central de Processamento - \textit{Central Processing Unit} \tabularnewline
API & Interface de Programação de Aplicações - \textit{Application Program Interface}\tabularnewline
RTOS & \textit{Real Time Operating System}\tabularnewline
ROS & \textit{Robotic Operating System}\tabularnewline
M3 & Sistema de Controle M3\tabularnewline
LARA & Laboratório de Automação e Robótica da UnB\tabularnewline
DSP & \textit{Digital Signal Processing}\tabularnewline
SEA & \textit{Serie Elastic Actuator}\tabularnewline
\end{tabular}


% Corpo Principal ----------------------------------------------------

\mainmatter
\setcounter{page}{1}
\pagenumbering{arabic}
\pagestyle{plain}

% Introdução
% - Descrição e contextualização do problema
% - Método de investigação utilizada
\chapter{Introdução} \label{ch:intro}

% Resumo opcional. Comentar se não usar.
%\resumodocapitulo{Resumo opcional}

\section{Contextualização}

Ao longo de toda a história humanidade, máquinas foram desenvolvidas para reduzir os esforços necessários para execução qualquer tarefa. Inicialmente foram voltadas a produção bens de consumo, passando a todo tipo de produção inclusive de máquinas que produzem outras máquinas já totalmente integradas ao cotidiano como exemplo máquinas para fabricar desde coisas mais simples como martelo e pregos até mais complexas como computadores e celulares.

E particular os robôs industriais dentro os vários tipos de máquinas existentes vêm ganhando cada vez mais espaço na área de produção e pouco a pouco integrando outros espaços como escolas, ateliês e hospitais. São definidos oficialmente pela ISO como "manipuladores multipropósito controlados automaticamente, reprogramáveis, programáveis em três ou mais eixos". Desta forma são máquinas extremamente flexíveis permitindo o emprego em uma diversidade de tarefas e cenários e garantido extrema consistência e precisão de forma automática.

No entanto, as mesmas características que conferem a robustez apreciada no ambiente industrial tornam robôs perigosos. A força, peso e velocidade de operação compõe elementos de muito risco na execução de atividades na presença de pessoas. Para resolver este problema novos tipos de robôs industriais têm sido desenvolvidos incorporando característica de segurança em cada aspecto do design, entre elas peso menor, capacidade de operação em velocidades reduzidas, protocolos para interrupção em caso de acidentes e complacência.

Muito embora ainda exista dentro do imaginário popular a visão da robótica como grande vilão no roubo de empregos, o maior esforço recente é em trabalhar na cooperação entre máquinas e pessoas para facilitar cada vez mais a vida de todos. Desta forma a introdução de elementos para segurança trazer os benefícios da robótica também a outros ambientes fora da industria sem a necessidade de isolamento. Conferindo uma combinação entre a alta capacidade de repetibilidade e precisão dos robôs junto a habilidade humana de rápida aprendizagem na hora de lidar com problemas ainda mais complexos.

Uma nova arquitetura do robô completamente pensada em segurança traz também uma maior dificuldade no controle. A ciência em torno do controle de sistemas rígidos está bem consolidada. Porém ainda são necessários estudos para garantir que robôs possam ser precisos e complacentes ao mesmo tempo. Isto é que possuam certa maleabilidade em caso de colisão para não causar danos sem deixar de permitir a execução da tarefa.

% Garantir flexibilidade e precisão: Tenho que ensinar kung fu ao robô...

% Robótica Cooperativa
% https://medium.com/@abhasvc/ais-threat-to-society-is-scarier-than-trump-ff7e9d42ea74
% https://www.hhs.se/contentassets/c8f677a0c9974bde950e2cec2edc51a1/substitution-of-labor-final.pdf

\section{Definição do problema}

No Laboratório de Automação e Robótica ( LARA - UnB ) encontra-se disponível o braço robótico Meka A2 fornecido pela Meka Robotics. Daqui em diante será denominado apenas Meka para simplificar. O Meka é um braço complacente antropomórfico composto por 7 juntas e uma garra desenvolvido para pesquisas na área de interação com pessoas. Cada uma das juntas possui um atuador série elástico composto de um motor brushless e um redução mecânica por onda de deformação.

Em trabalhos anteriores foi observado um desempenho ineficiente nos controles cinemáticos implementados em contraste ao resultado observado em outras plataformas robóticas e em simulação. Tal arquitetura não encontra-se devidamente documentada e detalhada. Desta forma, neste trabalho, propõe-se realizar uma caracterização mais detalhada desta arquitetura de controle para compreender suas implicações no desempenho dos controles cinemáticos testados e propor melhorias que resultem em trajetórias mais precisas levando as características próprias do sistema.

\section{Metodologia}

\subsection{Trabalhos Anteriores}

Inicialmente foi feito um estudo sobre os trabalho anteriores desenvolvidos na plataforma no Laboratório de Automação e Robótica ( LARA - UnB ) bem como estratégias clássicas no controle de robôs industriais. Alguns destes trabalhos puderam ser acompanhados ao longo da execução como o caso dos trabalhos desenvolvidos por Marcos Pereira e pelo Rafael Koji de modo a facilitar a transmissão do conhecimento relacionado as contribuições de cada um ao projeto bem como presenciar as dificuldades relacionadas ao controle do braço robótico.

\subsection{Metodologia de Investigação}

Após o estudo preliminar foi adotado uma metodologia de investigação baseada em ciclos compostos por 3 etapas, descritas a seguir. Estas etapas são baseadas nos princípios de desenvolvimento ágil e têm como objetivo acelerar e otimizar os esforços.

% Diagrama das etapas
\begin{enumerate}
    \item Análise e Formulação de Hipóteses
    \item Formulação de Testes para as Hipóteses
    \item Avaliação de Hipóteses ( Testes Experimentais e Estudo teórico )
\end{enumerate}

Cada ciclo possui duração variável de acordo com o nível de aprofundamento necessário para satisfazer os objetivos levantados para cada momento do projeto. O objetivo central é ao final de cada ciclo trazer algum aprimoramento quanto ao entendimento do sistema ou ainda quanto ao comportamento final na realização da tarefa.

No inicio de cada ciclo é feito um estudo do sistema através da modelagem do sistema e da definição de uma métrica para avaliar o comportamento. Então o comportamento do sistema real é comparado com o modelo adotado e as expectativas de comportamento. Os objetivos são traduzidos em uma métrica, compondo um conjunto de indicadores que permita avaliar se tarefa foi executada e como foi o processo de execução. A cada novo aspecto avaliado podem ser introduzidos novos indicadores em conjunto aos antigos ou modificados. 

Com base nisto são levantadas hipóteses quanto ao comportamento quanto ao fato de satisfazer ou não o modelo adotado e possíveis condutas para aprimorar os resultados dentro da métrica proposta. Para cada uma destas condutas é levantado um teste de verificação que pode incluir consulta bibliográfica, simulações a partir do modelo e experimentos com o sistema. Os resultados desta etapa são então levados para a etapa de observação e confrontados novamente e assim o ciclo se repete.

Resultados positivos são incorporados e resultados negativos avaliados quanto a possíveis correções. Tudo é registrado para permitir futuras avaliações. O resultado de cada dia foi feito um registro das atividades e todo material produzido foi colocado no GitHub de forma a permitir o uso futuro na validação do comportamento após modificações futuras.

Desta forma, os resultados da investigação ao longo do período deste trabalho compõe os capítulos  \ref{ch:tech-reference} e \ref{ch:results} aonde são apresentados os aspectos técnicos de cada componente analisado e resultados dos testes efetuados. 

% Estudo em malha fechada


% Fundamentos
% - Teoria por trás do problema
% - Tecnologias por trás do problema
\chapter{Fundamentos Teóricos}\label{ch:teory-reference}

% Quote: Sobre o ombros de gigantes

% Resumo opcional. Comentar se não usar.
%\resumodocapitulo{Resumo opcional}

A primeira etapa de qualquer processo na engenharia é modelagem. A partir dela relacionamos todos conhecimentos relacionados a um sistema em busca de avaliar o comportamento através de uma descrição mais simplificada. Uma das principais formas é a modelagem matemática, em que é descrito as relações lógicas entre cada parte do sistema,. E assim podemos através de operações matemáticas analisar em profundidade bem como extrapolar o estudo para sistema com modelos parecidos.

O manipulador é modelado como uma série de barras ligadas entre si, cada qual aproximada como um corpo rígido. Assim o movimento completo do manipulador é descrito a partir do conjunto dos movimentos de cada uma das  um corpo rígido pode ser descrito de forma cinemática ou dinâmica. 

% o que é efetuador, junta ?

\section{Modelagem Cinemático}

Na abordagem cinemática estamos interessado apenas no movimento em si, independente das forças que o causaram. Desta forma são observados a posição, velocidade e aceleração de cada parte. Em um manipulador robótico, de forma geral, os movimentos são escolhidos em busca da resolução de uma tarefa pela localização de alguma ferramenta acoplada ao braço. Estas por sua vez são posicionadas no última junta por ser o ponto de maior alcance.

% O que é uma cadeia cinemática ?

Cada junta é controlada de maneira separada, assim para o controle da ponta é preciso converter o configuração das juntas somadas ao modelo geométrico das partes do braço para uma orientação e posição do efetuador no espaço em relação a base do manipulador.

% Cinemática


\subsection{Descrição DH}

Cada barra do braço, pode ser descrita a partir dos dois pontos em cada uma das extremidades. Por sua vez cada um dos pontos é definido por 6 parâmetros, 3 para a posição em relação aos eixos x,y e z e 3 para orientação a partir da rotação em relação a cada um dos eixos. Somando 12 parâmetros para cada barra. 

Como forma de reduzir a quantidade de parâmetros necessários para cada link é adotado uma convenção baseada em apenas 4 valores denominada parâmetros de Denavit-Hatemberg. Cada barra é descrita a partir de 2 valores de distância e dois ângulos. Esta redução leva em conta que as partes do braço estão sempre conectadas e por tanto a descrição a partir dos pontos nas extremidade traria sempre redundância nos valores. Assim é descrito a relação entre cada um dos pontos através das medidas da distância e orientação dos encaixes nas extremidades de cada barra.

% Ilustração DH

% Configuração de junta?
Estes parâmetros são definidos de tal forma que apenas um possui variação com o acionamento. Em juntas prismáticas este representa variação da distância enquanto em enquanto em juntas cilíndricas um do ângulos que variam. Quando ocorre mais 
O registro de posição atual de cada junta representa a configuração da junta. 

A partir da descrição é feito várias operações de mudança de base para relacionar a posição e orientação do efetuador em relação ao ponto da base. Para cada uma das juntas do braço é avaliado a variação da orientação e da posição em relação a extremidade da junta anterior. A partir da combinação destas operações em cadeia é possível determinar uma expressão relacionando a configuração de cada junta com a posição final

\subsection{Descrição Vetorial}

\subsection{Matrizes Homogenias}

\subsection{Quatérnions Duais}
\subsubsection{Números Complexos}
\subsubsection{Quatérnions}
\subsubsection{Números Duais}
\subsubsection{Representação por Quatérnions Duais}

A representação em quatérnions duais traz como vantagens: eliminar a singularidades, inexistência de ambiguidade de representação. Além de trazer uma representação mais compacta e portanto computacionalmente mais eficiente, uma vez que são utilizados 8 parâmetros ao invés de 12. % Citar paper IROS Bruno Adorno

São utilizados dois quatérnions, sendo um para a representação da posição e um quatérnion unitário para representação da orientação. 

\section{Controle}



% Definição de Controle

\subsection{Controle PID}

\subsection{Controle em Cascata}

% Desenvolvimento
\chapter{Desenvolvimento\label{ch:develop}}

\resumodocapitulo{Simple, clear purpose and principles give rise to complex, intelligent behavior. Complex rules and regulations give rise to simple and stupid behavior.(Hock Dee)}

\section{Desempenho de Sistemas Mecatrônicos}

Em trabalhos anteriores foram propostos e implementados vários controladores cinemáticos como forma de corrigir o efeito das pertubações devido a interação com uma agente externo e da ação da gravidade. No entanto o desempenho foi abaixo do apresentado pela código de demonstração do fabricante para o robô e pelos mesmos controladores no uso em outra plataforma explicitando a necessidade de um maior estudo da arquitetura do manipulador robótico Meka A2. Com base nisto foi feito uma análise do componentes do braço e a implementação dos sistemas de controle das juntas em software e hardware. Neste capítulo será descrito os métodos de investigação utilizados bem como experimentos efetuados.

Em \cite{marcosps2016} Marcos propôs diversas métricas e avaliar os controladores cinemáticos implementados. Enquanto foi possível um excelente detalhamento da desempenho comparativa de cada um dos controladores, foi descoberto limitações na interação com a plataforma. Neste trabalho será analisado o robô enquanto sistema observando as implicações dos componentes mecânicos, sistemas embarcados e sistemas de controle operando de forma conjunta. O objetivo é avaliar dentro do que já foi implementado os limites de desempenho permitidos pela plataforma para permitir um melhor desempenho em trabalhos futuros.

\begin{figure}[H]
    \centering
    \includegraphics[width = 0.9\linewidth]{tex/figs/system_perfomance.png}
    \caption{Representação do Desempenho de um sistema \cite{paine2014high}}
    \label{fig:system_perfomance}
\end{figure}

Por se tratar de um sistema mecatrônico, naturalmente existem 3 camadas: Sistema Mecânico, Sistema Embarcado e Algorítimos de Controle conforme ilustrado na figura \ref{fig:system_perfomance} proposta por N. A. Paine em \cite{paine2014high} para o desenvolvimento de um sistema série elástico de alta performance. Para o Meka estas camadas estão implementados na seguinte forma:

\begin{itemize}
    \item Sistema Mecânico ( Atuadores Série Elásticos )
    \begin{itemize}
        \item Motores Brushless
        \item Harmonic Drive
    \end{itemize}
    \item Sistema Embarcado ( Atuadores Série Elásticos )
    \begin{itemize}
        \item Placa DSP para controle dos motores
        \item Hub EtherCAT
        \item Interface EtherCAT para o computador
        \item PC com Linux Ubuntu 12.04 e Kernel RTOS Xenomai
    \end{itemize}
    \item Algoritmos de Controle
    \begin{itemize}
        \item Controle de Torque (DSP)
        \item Controle de Posição com Compensação da Gravidade (PC)
        \item Interface ROS
        \item Controle Cinemático
        \item Gerador de Trajetória
    \end{itemize}
\end{itemize}

Como forma de alcançar um maior desempenho com o Meka foram avaliadas cada parte quanto a possíveis limitações e assim orientar trabalhos futuros no controle do robô.

% Comentar sobre desempenho de sistemas
% - Como alcançar o maior desempenho do sistema ( Custo exponencial 99.9 )
% - Não linearidades identificadas

% Avaliação Sistema Mecânico
% - Estudo SEA
% - Documentação
% Avaliação Sistema Embarcado?
% - Documentação EtherCAT, RTOS, PC
% - Tempo de Resposta da Comunicação
% Avaliação Controle
% - Ensaios Estabilidade em Malha Fechada
% - Ensaios em Degrau

% TODO Colocar nos resultados:
%No qual foi observado que aspectos particulares da plataforma não foram levados em conta como a operação em regiões não lineares do sistema devido a saturação da velocidade e torque dos motores, tempos longo de atraso de comunicação entre o computador.

% Reprodução Resultados Marcos

\section{Avaliação Preliminar}

Inicialmente foi feito um estudo a partir do código de demonstração do fabricante feito em Python para avaliar quais as possíveis forma de controlar o braço. Em que foi levantando todos os controladores implementados na biblioteca m3 e testados individualmente através da API em Python. Foram analisados os seguintes controladores disponíveis pelo M3:

\begin{itemize}
    \item Controle de Posição
    \item Controle de Posição com compensação da gravidade
    \item Controle de Torque
    \item Controle de Torque com compensação da gravidade
    \item Controle de Velocidade
\end{itemize}

%Nesta avaliação foi notado que alguns dos controladores da m3 não estavam disponíveis nas interfaces em C++. 

Para avaliar o comportamento em conjunto dos controladores cinemáticos apenas o controlador de posição com compensação da gravidade foi utilizado, uma vez que este que é usado pela ROS.

\section{Estudo Controladores Cinemáticos}

Para definir um ponto de referência para os ensaios e testes, foram avaliados os controladores implementados por Marcos Pereira. A partir da melhor configuração para cada um dos controladores foram feitos experimentos ajustando os parâmetros da velocidade de atuação e nível rigidez para avaliar a influência nos resultados. Os experimentos foram executados foram feitos com base em duas trajetórias pré-definidas: deslocamento em linha reta na vertical e o desenho de um quadrado a partir de dois deslocamentos na vertical e dois na horizontal.

Nos experimentos de deslocamento em linha reta foram avaliados o intervalo de tempo até robô começar a responder e o controle estabilizar para diferentes taxas de amostragem da trajetória. Enquanto nos experimentos com a trajetória de quadrado foram avaliadas a resposta do robô para diferentes valores de ajuste dos controladores de junta quanto aos parâmetros de velocidade e rigidez.

%\subsection{Avaliação tempo de amostragem}

O estudo dos tempos de atuação foi obtido repetindo o mesmo experimento para o deslocamento em linha reta na vertical em diferentes condições de velocidade e rigidez. Neste experimento foi estudado o tempo necessário para o atuador começar a se mover dado um comando bem o tempo necessário para o controle convergir.

% Resultados

%\subsection{Estudo Interação do controle de rigidez}

\section{Layout Experimentos}

Como forma de avaliar a resposta do sistema para controle dos ângulos de junta foi proposto um teste com o uso de um degrau de referência de entrada em malha fechada. Este experimento foi efetuado utilizando as interfaces da M3 definidas pela API em Python e pelo ROS. Para ambos casos os dados obtidos foram registrados em arquivo utilizando a ferramenta rosbag com auxílio do nó \textit{shm\_humanoid\_interface}.

% Comentar sobre como a informação é publicada nos tópicos
% Levantamento dos problemas

% Investigação de possíveis causas

\subsection{ROS}

Robotics Operating System (ROS) é um \textit{middleware} desenvolvido para a robótica. Desenvolver um robô é uma tarefa muito complexa para uma pessoa só ou mesmo um grupo de pesquisa, de modo que o ROS atua conectando diversos \textit{frameworks}, bibliotecas permitindo o uso de uma variedade de tipos de hardware e linguagens de programação \cite{quigley2009ros}. De igual forma atua conectando toda a comunidade de robótica em uma linguagem comum. Seu desenvolvimento começou em 2007 reunindo conceitos de diversos projetos de software aberto existentes até então e com o passar dos anos se tornou um padrão dentro da comunidade, contando com implementação para diversos robôs comerciais e inclusive uma versão completa voltada para a industria.

Como exemplo temos o módulo \textit{MovitIt} ( figura \ref{fig:movit-baxter} ) que reúne varias aplicações para facilitar o planejamento de trajetórias bem como a definição dos parâmetros de ajuste dos controladores com auxílio de métodos heurísticos através da biblioteca OMPL \cite{openMPL}.

\begin{figure}[H]
    \centering
    \includegraphics[width=0.7\linewidth]{tex/figs/movit-ros-baxter.png}
    \caption{Planejamento de Trajetória no Baxter com Auxílio do MoveIt \cite{coleman2014reducing}}
    \label{fig:movit-baxter}
\end{figure}

\subsection{shm\_humanoid\_interface}

Este é um nó do ROS disponibilizado pela Meka Robotics para a comunicação via ROS com o Meka. Representa uma das possíveis interfaces com a M3 através do acesso direto a memória compartilhada. em \ref{fig:mekaarch} são detalhados todas as possíveis interfaces. Através deste nó são gerados dois tópicos no ros: $/humanoid\_state$ e $/humanoid\_command$. A cada interação a informação atual do tópico $/hummanoid\_command$ é passada para a memória compartilhada e a leitura dos sensores é então escrita no tópico $/hummanoid\_state$. Na versão atual são somente interpretados os comando de posição das juntas através dos modos de controle com e sem compensação da gravidade.

\subsection{Rosbags}

Rosbags\footnote{\url{https://wiki.ros.org/rosbag}} é um utilitário do ROS para o registro dos eventos em cada tópico. A informação publicada nos tópicos é registrada em formato \textit{yaml} e permite que que posteriormente possa ser reproduzida em tópicos tal uma simulação. Desta forma dispensa a necessidade de manter o hardware conectado para analisar o comportamento do robô ao longo do tempo bem como permitir a análise de diferentes experimentos diretamente pelas ferramentas do ROS como o \textit{ros\_plot}. Ou ainda, permite conectar os resultados de um hardware real com uma simulação como forma de efetuar uma validação cruzada do comportamento real em comparação do comportamento simulado. Para tal basta executar o programa e indicar quais tópicos serão monitorados. Ao final da execução é registrado um arquivo com as informações com a extensão $.bag$. Internamente os dados são registrados em formato $.yaml$ o que permite a leitura direta.

Neste trabalho todos os experimentos foram registrados usando rosbag para permitir o uso futuro por outras pessoas. No entanto, como a representação da informação é um pouco verbosa gerando um arquivo grande, foi utilizado o programa $bag2csv$ para converter para o formato $.csv$ para facilitar o processamento. Em seguida o programa $sed$ para pequenos ajustes e a separação da leitura dos tópicos  $/humanoid\_state$ e $/humanoid\_command$ em diferentes arquivos. Este processo foi incorporado a um Makefile para permitir o processamento rápido de todos os arquivos $.bag$ utilizando o recurso de processamento paralelo da ferramenta \textit{GNU make}\footnote{\url{https://www.gnu.org/software/make/manual/html_node/Parallel.html}}.

\subsection{Python API}

A M3 possui uma interface através de uma API em Python que permite uma liberdade maior de controle uma vez que possui mais recursos já implementados. Além dos modos de controle por posição estão também disponíveis o controle direto por torque e por velocidade das juntas e o controle por PWM dos motores. Na M3 estão implementados um controlador para compensação da gravidade para o controle de torque e de posição a partir da biblioteca KDL com base no momento dinâmico do braço descrito no arquivo $m3ene/meka_doc/ma26/m3dynamatics\_right\_ma26.yml$. Este arquivo pode ser alterado para permitir o controle com um atuador com peso diferente, sem ter que alterar os parâmetros dos controladores.

% Explicação sobre funcionamento da biblioteca
% Diagrama: Comando/Estado -> Memória "bot" -> "proxy.step()"
% -> Problema Detectado no uso do ângulo em radianos por isto foi adotado o uso em graus

% Resultados
\chapter{Resultados\label{chap:FundamentacaoMatematica}}

% Resumo opcional. Comentar se não usar.
\resumodocapitulo{Resumo opcional.}

\section{Estudo Arquitetura}

% Resumo informações da arquitetura relevantes do ponto de vista de controle

\section{Estudo Controladores Implementados}

\subsection{DQ Robotics}

\subsection{API Python}

Introduzir.


% Conclusão
\chapter{Conclusão} \label{ch:conclusoes}

\resumodocapitulo{``"What we need is more people specialize in the impossible '' (Theodore Roethke)}

\section{Perspectivas Futuras}

% Introdução 
% - Cobots
Neste trabalho foi apresentado um detalhamento maior da arquitetura de controle implementada no Meka permitindo conhecer os fundamentos das tecnologias envolvidas no desenvolvimento de manipuladores robóticos cooperativos bem como vivenciar os principais desafios em manter um manipulador robótico seguro e um desempenho razoável na execução de alguma tarefa. Foi buscado ir a origem inicial de cada problema, tanto do ponto de vista histórico como de cada componente utilizado. O redução no preço dos computadores, o uso de sistemas série elásticos para controle preciso de torque e bem como a disponibilidade de soluções open-source entre outras tecnologias permitiram que manipuladores robóticos pudessem alcançar mais pessoas. Soluções robóticas sempre estiveram no plano de tecnologias de alto padrão e distantes da realidade da maioria porém têm sido crescente os esforços na difusão.

% Fundamentação
% - Qualidade do HW
O Meka foi implementado com componentes de alto desempenho seguindo a filosofia do mercado na sua época de lançamento. Embora hoje, outras tecnologias estão sendo incorporadas ao projeto de robôs complacentes como o uso de sistemas freios, esta plataforma ainda representa uma ótima plataforma para o estudo de controladores dinâmicos bem como cada uma dos componentes envolvidos. Como exemplo, o uso de atuadores série elásticos, embora estejam implementados no Meka com motores e sistemas de transmissão de ponta, podem também ser incorporados em outros projetos como solução aos problemas de instabilidade do controle na interação com superfícies rígidas em aplicações para locomoção. Ou ainda em aplicação de forças na área de fisioterapia.

% Desenvolvimento
No que tange ao sistema de controle implementado, foi percebido que nos trabalhos anteriores foi alcançado o limite de desempenho dos controladores de baixo atuais do sistema M3. No entanto, dado que o sistema de controle é completamente open-source e possui um sistema embarcado capaz de uma desempenho maior ainda existe um enorme potencial que pode ser explorado para uso em trabalhos futuros. Uma vez que as soluções implementadas incorporam técnicas de filtragem e controle dinâmico que podem ser aprimoradas.

% Resultados
% - Juntas
Os controladores de posição de juntas foram ajustados para responderem de forma similar para todas as juntas. Embora esta seja uma estratégia que permite que a resposta controle seja similar e facilite a avaliação dos controladores cinemáticos, na prática o comportamento o comportamento de cada junta é bem diferente. Para um ganho de desempenho, pode ser explorado a possibilidade de controladores ajustados para respostas diferente para cada uma das juntas. Em conjunto com as características particulares da geometria do braço, um desempenho melhor pode ser atingido, por permitir as juntas do ombro operarem com uma resposta mais rápida.

A técnica atualmente implementada para compensação da gravidade usa apenas a posição das juntas para uma gerar um torque em feedforward a partir do modelo cinemático e dinâmico do robô. Outras técnicas podem ser utilizadas para que as velocidade final e o torque possam também ajustadas. Também foi notado que o acoplamento entre a posição e a orientação introduzido por quatérnions duais pode ser explorado a partir das características das juntas do robô, uma vez que a composição do ombro e do pulso podem atuar como dois guibal em oposição permitindo um estudo maior da dinâmica da rotação no movimento do efetuador.

% Trabalhos Futuros Controle + Metodologia
Para trabalhos futuros na plataforma é sugerido a identificação do modelo dos atuadores série elástico disponíveis na plataforma visando a implementação de um controle de juntas mais preciso. No intuito de apresentar a modelagem completa do braço, uma vez que esta não está disponível. Também é apresentado a possibilidade de estudo em controle multivariado a partir de espaço de estados para permitir explorar os efeitos das não linearidade em composição com o modelo cinemático do robô, gerando uma solução que controle ao mesmo a cinemática e dinâmica do movimento. Os conceitos implementados na plataforma podem também compor partes de outros projetos com explorado por outras universidades que passaram a produzir seus próprios atuadores, braços e plataformas robóticas.

%Com base nos resultados obtidos, observou-se o controle implementado em ROS está sendo executado com um frequência maior que a resposta das camadas inferiores subsequentes de controle. Este fator pode estar causando uma instabilidade no controle, uma vez que o erro acumulado durante o tempo que o robô permanece parado até o ocorrer a resposta é compensado com um esforço maior de controle, percebido nos registros dos experimentos como saltos periódicos na velocidade. Estes efeitos eram percebidos com maior nitidez no eixo vertical, uma vez que a maior perturbação ao sistema se dava pelo efeito da ação do campo gravitacional.

%Como proposta, serão efetuados ao longo do trabalho a continuidade do estudo da plataforma em conjunto com a identificação da planta completa. Por se tratar de um sistema complexo que envolve diversos sub-sistemas atuando em conjunto, este estudo será pautado as principais influências de cada componente no controle final da posição do efetuador.

%Também serão analisado um comparativo das estratégias de controle por velocidade e por posição a partir da descrição cinemática implementada em Quatérnions Duais. O controle por velocidade apresenta uma resposta mais rápida que os controles de velocidade por retirar a necessidade e esperar a posição final ser atingida e eliminaria a necessidade do passo de integração por parte de alguns dos controladores implementados. Ao passo que poderia permitir o uso direto de controladores mais baixos, uma vez que os motores Brushless possuem os controle de torque e velocidade que são convertidos em posição por meio da placa DSP.

%Ao final, o objetivo é trazer desempenho melhor em termos de precisão no controle do Meka e facilitar o uso deste no laboratório em projetos futuros por meio do aprofundamento na documentação e estudo iniciados em trabalhos anteriores. Em conjunto é apresentado um detalhamento das sistemas empregados para inspirar o uso em projetos futuros de sistemas robóticos desenvolvidos pelo laboratório, uma vez soluções similares baseadas em atuadores série elásticos estão presentes muitos projetos recentes.

% Bibliografia ------------------------------------------------------

\renewcommand{\bibname}{REFERÊNCIAS BIBLIOGRÁFICAS}
\addcontentsline{toc}{chapter}{REFERÊNCIAS BIBLIOGRÁFICAS}

\bibliographystyle{abnt-num}
\bibliography{relatorio}
\nocite{adorno2017robot}
\nocite{mmurilo-biorob}
\nocite{Adorno2011}
\nocite{figueredo:2013}
\nocite{sentis2007synthesis}
\nocite{pratt1997stiffness}
\nocite{iyer2012modeling}
%\nocite{abeyrathna2013simulation} % Cited
\nocite{book:siciliano:2009}
\nocite{book:spong:}
\nocite{murlow2014}
\nocite{lee2017generalization}
%\nocite{pratt1995series} % Cited!
% Citar Livro
% https://new.abb.com/process-automation/process-automation-service/reference-pages/single-loop-control-methods

% Anexos -------------------------------------------------------------
% \anexos
% \makeatletter
% % não retirar estes comandos
%  \renewcommand{\@makechapterhead}[1]{%
%   {\parindent \z@ \raggedleft \setfontarial\bfseries
%      \LARGE \thechapter. \space\space
%      \uppercase{#1}\par
%      \vskip 40\p@ }}
%\makeatother

% Anexo I: Descrição do CD
%\chapter{Descrição do conteúdo do CD}

\label{AnCD}

Este trabalho é open source e está disponível no Github através do link \url{https://github.com/akafael/undergrad-thesis}. Dentro do CD e da repositório estão as seguintes pastas:

\begin{itemize}
    \item thesis
    \begin{itemize}
        \item tex: Código fonte, figuras e tabelas do relatório
        \item src: Código utilizado para gerar as figuras
    \end{itemize}
\end{itemize}


%\refstepcounter{noAnexo}

% Anexo II: Programas Utilizados
%\include{anexo_Codigos}
%\refstepcounter{noAnexo}

\end{document}
